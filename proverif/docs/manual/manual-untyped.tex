\documentclass{article}

\usepackage[latin1]{inputenc}
\usepackage{amsmath}
\usepackage[a4paper,hmargin=1in,vmargin=1.25in]{geometry}  
\usepackage{url}

\newcommand{\nonterm}[1]{\langle\textrm{#1}\rangle}
\newcommand{\seq}[1]{\textrm{seq}\nonterm{#1}}

% --------------------------------------------------------------------- %
% Typesetting definitions:              Sample output:                  %
%                                                                       %
% \begin{defn}                                                          %
% \categ{M,N}{terms}\\          M, N ::=        terms           %
% \entry{x}{variable}\\                   x               variable      %
% \entry{M\ N}{application}\\             M N             application   %
% \entry{\lambda x.\ M}{abstraction}      \x.M            abstraction   %
% \end{defn}                                                            %
%                                                                       %
% This is a tabbing environment; the last entry should have no \\.      %
% --------------------------------------------------------------------- %

\newenvironment{defn}{\begin{tabbing}
  \hspace{1.5em} \= \hspace{.27\linewidth - 1.5em} \= \hspace{1.5em} \= \kill
%  \hbra\\[-.8ex]
  }{
%\\[-.8ex]\hket
  \end{tabbing}}
\newcommand{\entry}[2]{\>$#1$\>\>#2}
\newcommand{\clause}[2]{$#1$\>\>#2}
\newcommand{\categ}[2]{\clause{#1::=}{#2}}
\newcommand{\subclause}[1]{\>\>\>#1}

\begin{document}

\title{ProVerif\\
Automatic Cryptographic Protocol Verifier\\
User Manual for Untyped Inputs}

\author{Bruno Blanchet\thanks{This work was partly done 
while the author was at {\'E}cole Normale Sup�rieure, Paris and at 
Max-Planck-Institut f�r Informatik, Saarbr{\"u}cken.}\\
INRIA Paris-Rocquencourt, France}

\maketitle

\textbf{Warning!} This manual documents the untyped inputs of ProVerif
(untyped Horn clauses, untyped pi calculus). These input formats are no
longer actively developed. We recommend coding your protocols in the
typed pi calculus format, described in the file {\tt manual.pdf}.

\section{Introduction}

This manual describes the untyped input syntax and output of ProVerif. 
It does not describe the internal algorithms used
in the system. These algorithms have been described in various
research
papers~\cite{Blanchet2001,Abadi04c,Blanchet2002,Abadi03b,Abadi04f,Abadi04,Blanchet04,Blanchet04e,Allamigeon05,Blanchet05},
that can be downloaded on

\centerline{{\tt
http://www.di.ens.fr/\string~blanchet/publications/index.html}.}

The tool can take two formats as input. The first one is in the form
of Horn clauses (logic programming rules), and corresponds to the
system described in~\cite{Blanchet2001}. The second one is in the form
of a process in an extension of the pi calculus, described
in~\cite{Abadi04c}. In both cases, the output of the system is
essentially the same.

\section{Common remarks on the syntax}

Comments can be included in input files. Comments are surrounded by
{\tt (*} and {\tt *)}. Nested comments are not supported.

Identifiers begin with a letter (uppercase or lowercase) and contain
any number of letters, digits, the underscore character (\_),
the quote character ('), as well as accented letters of the ISO Latin 1
character set. Case is significant. Each input
system has a number of keywords that cannot be used as ordinary
identifiers.

In case of syntax error, the system indicates the character position
of the error (line and column numbers). Please use your text editor to find the
position of the error. (The error messages can be interpreted
by \texttt{emacs}.)

\section{Command-line options}

The syntax of the command-line is 
\begin{quote}
$\texttt{./proverif }\nonterm{options}\ \nonterm{filename}$
\end{quote}
where the $\nonterm{options}$ can be
\begin{itemize}

\item $\texttt{-in }\nonterm{format}$

  Choose the input format (\texttt{horn}, \texttt{horntype},
  \texttt{pi}, \texttt{pitype}). When the \texttt{-in} option is absent, 
  the input format is chosen according to the file extension, 
  as detailed below. The recommended input format is 
  the typed pi calculus, which corresponds to the option
  \texttt{-in pitype}, and is the default when the file extension is
  \texttt{.pv}. It is described in {\tt manual.pdf}.
  The other formats are no longer actively developed.
  Input may also be provided using the untyped pi calculus (option
  \texttt{-in pi}, the default when the file extension is
  \texttt{.pi}), typed Horn clauses (option \texttt{-in horntype}, the
  default when the file extension is \texttt{.horntype}), and untyped
  Horn clauses (option \texttt{-in horn}, the default for all other
  file extensions). This manual documents the untyped Horn clauses 
  and the untyped pi calculus input formats.

\item $\texttt{-out }\nonterm{format}$

Choose the output format, either \texttt{solve} (analyze the protocol)
or \texttt{spass} (stop the analysis before resolution, and output the
clauses in the format required for use in the Spass first-order
theorem prover, see \url{http://www.spass-prover.org/}). 
The default is \texttt{solve}. When you select
\texttt{-out spass}, you must add the option $\texttt{-o }\nonterm{filename}$
to specify the file in which the clauses will be output.

\item $\texttt{-TulaFale }\nonterm{version}$

For compatibility with the web service analysis tool TulaFale
(see the tool download at \url{http://research.microsoft.com/projects/samoa/}).
The version number is the version of TulaFale with which you would
like compatibility. Currently, only version 1 is supported.

\item $\texttt{-color}$

Display a colored output on terminals that support ANSI color codes.
(Will result in a garbage output on terminals that do not support these codes.)
Unix terminals typically support ANSI color codes. For \texttt{emacs} users,
you can run ProVerif in a shell buffer with ANSI color codes as follows:
\begin{itemize}
\item start a shell with \texttt{M-x shell}
\item load the \texttt{ansi-color} library with \texttt{M-x load-library RET ansi-color RET}
\item activate ANSI colors with \texttt{M-x ansi-color-for-comint-mode-on}
\item now run ProVerif in the shell buffer.
\end{itemize}
You can also activate ANSI colors in shell buffers by default by
adding the following to your \texttt{.emacs}:
\begin{verbatim}
(autoload 'ansi-color-for-comint-mode-on "ansi-color" nil t)
(add-hook 'shell-mode-hook 'ansi-color-for-comint-mode-on)
\end{verbatim}

\item $\texttt{-help}$ or $\texttt{--help}$

Display a short summary of command-line options.

\end{itemize}



\section{Input as Horn clauses}\label{sect:horninput}

By default, the executable program \texttt{proverif} takes Horn
clauses as input. You can run it as follows:
\begin{quote}
$\texttt{./proverif }\nonterm{filename}$
\end{quote}
where $\nonterm{filename}$ references a file containing the Horn
clauses, in the format explained below. The system then basically
determines whether a fact can be derived from the clauses. If true, a
proof is given. As shown in~\cite{Blanchet2001}, this can be used to
determine secrecy properties of protocols: if a certain fact cannot be
derived from the clauses, then the secrecy of a certain value is
preserved. A difference with first-order theorem provers that perform
a similar task, is that correctness and completeness are
reversed. Here, correctness means that if a value is not secret, the
system says so, that is, if a fact is derivable, the system says
so. Completeness means that if a fact is not derivable, then the
system says so. We sometimes drop completeness (that is, we lose
precision; see options below), but never correctness.

The keywords of this input system are 
{\tt data}, {\tt elimtrue}, {\tt equation}, {\tt fun}, {\tt not}, {\tt nounif}, {\tt
param}, {\tt pred}, {\tt query}, and {\tt reduc}.


\begin{figure}[t]
\begin{align*}
&\nonterm{clause} ::= [( \nonterm{fact}
\texttt{ \& })^*  \nonterm{fact}\texttt{ ->}]\ \nonterm{fact}\\
&\phantom{\nonterm{clause}}\mid\phantom{:=} ( \nonterm{fact}
\texttt{ \& })^*  \nonterm{fact}\texttt{ <-> } \nonterm{fact}\\
&\phantom{\nonterm{clause}}\mid\phantom{:=} ( \nonterm{fact}
\texttt{ \& })^*  \nonterm{fact}\texttt{ <=> } \nonterm{fact}\\
&\nonterm{fact} ::=  \nonterm{ident} \texttt{:} \seq{term}\\
&\phantom{\nonterm{fact}}\mid\phantom{:=}\nonterm{term}\texttt{ <> }\nonterm{term}\\
&\nonterm{term} ::= \nonterm{ident}\texttt{(}\seq{term}\texttt{)}\\
&\phantom{\nonterm{term}}\mid\phantom{:=}\nonterm{ident}\texttt{[}\seq{term}\texttt{]}\\
&\phantom{\nonterm{term}}\mid\phantom{:=}\texttt{(}\seq{term}\texttt{)}\\
&\phantom{\nonterm{term}}\mid\phantom{:=}\nonterm{ident}\\
&\nonterm{factformat} ::=  \nonterm{ident} \texttt{:} \seq{termformat}\\
&\nonterm{termformat} ::= \nonterm{ident}\texttt{(}\seq{termformat}\texttt{)}\\
&\phantom{\nonterm{termformat}}\mid\phantom{:=}\nonterm{ident}\texttt{[}\seq{termformat}\texttt{]}\\
&\phantom{\nonterm{termformat}}\mid\phantom{:=}\texttt{(}\seq{termformat}\texttt{)}\\
&\phantom{\nonterm{termformat}}\mid\phantom{:=}\nonterm{ident}\\
&\phantom{\nonterm{termformat}}\mid\phantom{:=}\texttt{*}\nonterm{ident}
\end{align*}
where $\seq{X}$ is a sequence of $X$: $\seq{X} = [(\nonterm{X}\texttt{,})^*\nonterm{X}] = \nonterm{X}\texttt{,}\ldots\texttt{,}\nonterm{X}$. (The sequence can be empty, it can be one element $\nonterm{X}$, or it can be several elements $\nonterm{X}$ separated by commas.)
\caption{Grammar of facts and clauses}\label{fig:rsyntax}
\end{figure}

The input file consists of a list of declarations, followed by the
keyword {\tt reduc} and a list of clauses:
\[\nonterm{declaration}^*\ {\tt reduc}\ (\nonterm{clause}\texttt{;})
^*\ \nonterm{clause}\texttt{.}\]
The syntax of facts and clauses is given in Figure~\ref{fig:rsyntax}. In
this grammar $X^*$ means any number of repetitions of $X$, $[X]$ means
$X$ or nothing. Text in typewriter style should appear as it is in
the input file. Text between $\langle$ and $\rangle$ represents
non-terminals.

Declarations can be any of the following:
\begin{itemize}

\item $\texttt{param }\nonterm{name}\texttt{ = }\nonterm{value}\texttt{.}$

This declaration sets the value of configuration parameters.
The following cases are supported:

\begin{itemize}

\item \texttt{param verboseRules = false.}\\
\texttt{param verboseRules = true.}

Display the number of clauses every 200 clause created during the solving
process (\texttt{false}) or display each clause created during the solving
process (\texttt{true}).

\item \texttt{param verboseRedundant = false.}\\
\texttt{param verboseRedundant = true.}

Display eliminated redundant clauses when \texttt{true}.

\item \texttt{param verboseCompleted = false.}\\
\texttt{param verboseCompleted = true.}

Display completed set of clauses after saturation when \texttt{true}.

\item \texttt{param verboseEq = true.}\\
\texttt{param verboseEq = false.}

Display information on handling of equational theories when \texttt{true}.

\item \texttt{param verboseTerm = true.}\\
\texttt{param verboseTerm = false.}

Display information on termination when \texttt{true}
(changes in the selection function to improve termination;
termination warnings).

\item \texttt{param maxDepth = none.}\\
$\texttt{param maxDepth = }n\texttt{.}$

Do not limit the depth of terms (\texttt{none}) or limit the depth of
terms to $n$, where $n$ is an integer. A negative value means no
limit. When the depth is limited to $n$, all terms of depth greater
than $n$ are replaced with new variables. Limiting the depth can be
used to enforce termination of the solving process at the cost of
precision.

\item \texttt{param maxHyp = none.}\\
$\texttt{param maxHyp = }n\texttt{.}$

Do not limit the number of hypotheses of clauses (\texttt{none}) or
limit it to $n$, where $n$ is an integer. A negative value means no
limit. When the number of hypotheses is limited to $n$, arbitrary
hypotheses are removed from clauses, so that only $n$ hypotheses
remain. Limiting the number of hypotheses can be used to enforce
termination of the solving process at the cost of precision
(although in general limiting the depth by the above declaration is
enough to obtain termination).

\item \texttt{param selFun = TermMaxsize.}\\
\texttt{param selFun = Term.}\\
\texttt{param selFun = NounifsetMaxsize.}\\
\texttt{param selFun = Nounifset.}

Chooses the selection function that governs the resolution process.
All selection functions avoid unifying on facts indicated by a
\texttt{nounif} declaration. \texttt{Nounifset} does exactly
that. \texttt{Term} automatically avoids some other unifications, to
help termination, as determined by some heuristics.
\texttt{NounifsetMaxsize} and \texttt{TermMaxsize} choose the fact of
maximum size when there are several possibilities. This choice
sometimes gives impressive speedups.

\item \texttt{param stopTerm = true.}\\
\texttt{param stopTerm = false.}

Display a warning and wait for user answer when the system thinks the
solving process will not terminate (\texttt{true}), or go on as if
nothing had happened (\texttt{false}). This setting applies only to
the selection functions \texttt{NounifsetMaxsize} and \texttt{Nounifset}.
(See parameter \texttt{selFun}.)

\item \texttt{param redundancyElim = simple.}\\
\texttt{param redundancyElim = no.}\\
\texttt{param redundancyElim = best.}

An elimination of redundant clauses has been implemented: when a clause
without selected hypotheses is derivable from other clauses without
selected hypothesis, it is removed. With \texttt{redundancyElim =
simple}, this is applied for newly generated clauses. With
\texttt{redundancyElim = no}, this is never applied. With
\texttt{redundancyElim = best}, this is also applied when an old clause
can be derived from other old clauses plus the new clause.

\item \texttt{param redundantHypElim = beginOnly.}\\
\texttt{param redundantHypElim = false.}\\
\texttt{param redundantHypElim = true.}

When a clause is of the form $H \wedge H' \rightarrow C$, and
there exists $\sigma$ such that $\sigma H \subseteq H'$ and $\sigma$
does not change the variables of $H'$ and $C$, then the clause
can be replaced with $H' \rightarrow C$ (since there are
implications in both directions between these clauses).

This replacement is done when \texttt{redundantHypElim =
true.}, or when \texttt{redundantHypElim = beginOnly.} and
$H$ contains a \texttt{begin} event.
Indeed, testing this property takes time, and slows down
small examples. On the other hand, on big examples, in particular
when they contain several \texttt{begin} events (or blocking facts),
this technique can yield huge speedups.

\item \texttt{param reconstructDerivation = true.}\\
\texttt{param reconstructDerivation = false.}

When a fact is derivable, should we reconstruct the corresponding
derivation? (This setting has been introduced because in some extreme
cases reconstructing a derivation can consume a lot of memory.)

\item \texttt{param simplifyDerivation = true.}\\
\texttt{param simplifyDerivation = false.}

Should the derivation be simplified by removing duplicate proofs
of the same \texttt{attacker} facts? 

\item \texttt{param abbreviateDerivation = true.}\\
\texttt{param abbreviateDerivation = false.}

When \texttt{abbreviateDerivation = true}, ProVerif defines symbols to
abbreviate terms that represent names $a[\ldots]$ before displaying
the derivation, and uses these abbreviations in the derivation. These
abbreviations generally make reading the derivation easier by reducing
the size of terms.

\item \texttt{param displayDerivation = true.}\\
\texttt{param displayDerivation = false.}

Should the derivation be displayed? 
Disabling derivation display is useful for very big derivations.

\item \texttt{param symbOrder = "$f_1 > \dots > f_n$".}

ProVerif uses a lexicographic path ordering in order to prove
termination of convergent equational theories. By default,
it uses a heuristic to build the ordering of function symbols
underlying this lexicographic path ordering. This setting
allows the user to set this ordering of function symbols.

\end{itemize}
In the above list, the default value is the first mentioned.
The system also accepts \texttt{no} instead of \texttt{false}
and \texttt{yes} instead of \texttt{true}.

\item $\texttt{fun }\nonterm{ident}\texttt{/}n\texttt{.}$

Declares a function symbol  $\nonterm{ident}$ of arity $n$.

\item $\texttt{data }\nonterm{ident}\texttt{/}n\texttt{.}$

$\texttt{data }f\texttt{/}n\texttt{.}$ declares a data function symbol $f$ of arity $n$.
Data function symbols are similar to tuples: the adversary can construct
and decompose them. The system implicitly adds the equivalence
$p:f(x_1, \ldots, x_n) \Leftrightarrow p:x_1 \wedge \ldots \wedge p:x_n$
for each predicate $p$ declared $\texttt{decompData}$.

\item $\texttt{equation }\nonterm{term}\texttt{ = }\nonterm{term}\texttt{.}$

$\texttt{equation }M_1\texttt{ = }M_2$ says that the terms $M_1$ and
$M_2$ are in fact equal. The function symbols in the equation should be
only already declared constructors. The treatment of equations is
still rather limited. The equation $f(x,g(y)) =
f(y,g(x))$, used for Diffie-Hellman key agreements, is known to
work. The system may not terminate when more complex equations are
entered. In the presence of both equations and inequality constraints,
the system is not complete (but still correct): the inequality
constraints are deemed true when the terms are syntactically
different, without taking into account the equations.

\item $\texttt{query }\nonterm{fact}\texttt{.}$

Indicates that the system should determine whether $\nonterm{fact}$ is
true or not. If $\nonterm{fact}$ contains variables, determine which
instances of $\nonterm{fact}$ are true.

\item $\texttt{nounif }\nonterm{factformat}[/n]\texttt{.}$

Modifies the selection of facts to be resolved upon, to avoid
resolving facts that match $\nonterm{factformat}$. For a fact
$\nonterm{fact}$ to match $\nonterm{factformat}$, $\nonterm{fact}$
must contain a variable when $\nonterm{factformat}$ contains one, and
any term when $\nonterm{factformat}$ contains \texttt{*} followed by a
variable name. The optional integer $n$ indicates how much we should
avoid resolution on facts that match $\nonterm{factformat}$: the greater
$n$, the more such resolutions will be avoided.

\item $\texttt{pred }\nonterm{ident}\texttt{/}n\ \seq{ident}\texttt{.}$

$\texttt{pred }p\texttt{/}n\ i_1, \ldots, i_n\texttt{.}$ declares a new
predicate $p$, of arity $n$, with special properties described by
$i_1, \ldots, i_n$. 

The following properties are allowed for $i_1, \ldots, i_n$:
\begin{itemize}

\item \texttt{block}:
Declares the predicate $p$ as a blocking predicate. 
Blocking predicates may appear in hypotheses or conclusions of
clauses, but not both for the same predicate. 

When they appear in hypotheses, 
instead of trying to prove facts containing
these predicates (which is impossible since no clause implies such
facts), the system collects hypotheses containing the blocking
predicates necessary to prove the queries. This is useful in
particular to prove authenticity~\cite{Blanchet2002}.

When they appear in conclusions, the system makes sure to really
know whether the fact $C$ in question is derivable, and does not return
clauses $H \texttt{ -> } C$ for which it is not clear whether $H$
is derivable or not.

\item \texttt{elimVarStrict}:
Tells the system to that $p\texttt{:}new\_name[i],\ldots, new\_name[i]$
holds. Then ProVerif removes the hypothesis $p\texttt{:}x_1, \ldots, x_n$
where $x_1, \ldots, x_n$ are variables that do not appear elsewhere in a clause.

\item \texttt{elimVar}:
Tells the system to that $p\texttt{:}new\_name[i],\ldots, new\_name[i]$
holds. Then ProVerif removes the hypothesis $p\texttt{:}x_1, \ldots, x_n$
where $x_1, \ldots, x_n$ are variables that do not appear elsewhere in a clause,
except possibly in inequality facts. Removing such hypotheses
$\texttt{attacker}\texttt{:}x$ is complete for proving secrecy or
authenticity, because there always exists a value of $x$ that makes
true both the inequality facts, and $\texttt{attacker}\texttt{:}x$. 
In general, however, removing $p\texttt{:}x_1, \ldots, x_n$ where
$x_1, \ldots, x_n$ may appear in inequalities leads to a sound
approximation.
(When it is important that no underivable facts are considered derivable,
the transformation is applied only when $x_1, \ldots, x_n$ do not appear 
elsewhere in the clause, as for \texttt{elimVarStrict}.)

\item \texttt{decompData}:
Adds the following clauses, where $p$ is a predicate of arity $m$ and $f$ is
any function symbol of arity $n$ declared $\texttt{data}$ or a tuple function.
\begin{align*}
&p\texttt{:}f\texttt{(}x_{11}\texttt{,}\ldots\texttt{,}x_{1n}\texttt{),} \ldots\texttt{,}f\texttt{(}x_{m1}\texttt{,}\ldots\texttt{,}x_{mn}\texttt{) -> } p\texttt{:}x_{1i}\texttt{,}\ldots\texttt{,}x_{mi}\\
\begin{split}
&p\texttt{:}x_{11}\texttt{,}\ldots\texttt{,}x_{m1}\texttt{ \& }\ldots\texttt{ \& }
p\texttt{:}x_{1n}\texttt{,}\ldots\texttt{,}x_{mn}\texttt{ -> }\\
&\qquad p\texttt{:}f\texttt{(}x_{11}\texttt{,}\ldots\texttt{,}x_{1n}\texttt{),} \ldots\texttt{,}f\texttt{(}x_{m1}\texttt{,}\ldots\texttt{,}x_{mn}\texttt{)}
\end{split}
\end{align*}
For example, when $p$ is a unary predicate and $f$ is a tuple function,
we add:
\begin{align*}
&p\texttt{:(}x_1\texttt{,}\ldots\texttt{,}x_n\texttt{) -> } p\texttt{:}x_i\\
&p\texttt{:}x_1\texttt{ \& }\ldots\texttt{ \& }
p\texttt{:}x_n\texttt{ -> }
p\texttt{:(}x_1\texttt{,}\ldots\texttt{,}x_n\texttt{)}
\end{align*}
for all $n$ and $i \in \{ 1, \ldots, n\}$. 
These clauses are treated in a specially optimized way, since they
are used in most protocols.

\item \texttt{decompDataSelect}: same as \texttt{decompData}, 
but allows the selection of facts
$p\texttt{:}x_1, \ldots, x_n$ while \texttt{decompData} does not.
(It is in general better not to select such facts, because it
leads to non-termination in the presence of data decomposition
clauses; there are however exceptions for some rare cases.)

\item \texttt{memberOptim}:
This must be used only when $p$ is defined by
\begin{align*}
&p:x,f(x,y);\\
&p:x,y \texttt{ -> } p:x,f(x',y).
\end{align*}
where $f$ is a data constructor. It turns on the following
optimization: $p':x \wedge p:M_1,x \wedge \ldots \wedge p:M_n,x$
where $p'$ is declared \texttt{decompData} and $p$ is
declared \texttt{memberOptim} is replaced with $p':x \wedge p':M_1
\wedge \ldots \wedge p':M_n$ when $x$ does not occur elsewhere (just
take $x = f(M_1, \ldots f(M_n, x'))$ and notice that $p':x$ if and
only if $p':M_1$, \ldots, $p':M_n$, and $p':x'$), or when the
clause has no selected hypothesis. In the last case, this introduces
an approximation.

The replacement is also possible when $x$ occurs in several predicates
declared \texttt{decompData}. However, when $x$ occurs in several
\texttt{memberOptim} predicates, the transformation may introduce
an approximation. (For example, consider $p_1$  and $p_2$ defined
as above respectively using $f_1$ and $f_2$ as data constructors.
Then $p_1:M,x \wedge p_2:M',x$ is never true: for it to be true,
$x$ should be at the same time $f_1(\_,\_)$ and $f_2(\_,\_)$.)

\end{itemize}

\item $\texttt{elimtrue }\nonterm{fact}\texttt{.}$

The declaration $\texttt{elimtrue }F$ means that $\sigma F$ is true
for all $\sigma$; the fact $F$ is added to the set of clauses.

Then in a clause $R = F' \texttt{ \& }H\texttt{ -> }C$, if $F'$
unifies with $F$ with most general unifier $\sigma_u$ and all
variables of $F'$ modified by $\sigma_u$ do not occur in the rest of
$R$ then the hypothesis $F'$ can be removed: $R$ is transformed into
$H\texttt{ -> }C$, by resolving with $F$.

\item $\texttt{not }\nonterm{fact}\texttt{.}$

Adds a secrecy assumption, saying that $\nonterm{fact}$ cannot be
proved from the clauses. Then the system can remove all clauses that
contain fact in their hypotheses. (These clauses can never be applied.)
This speeds up the system. At the end of the solving process, the
system checks that $\nonterm{fact}$ can indeed not be derived from the
clauses. If it can be derived, the proof fails with an error message.

\end{itemize}
Two kinds of functions may appear in terms: constructors and names.
Constructors are followed by their parameters between parentheses:
$f\texttt{(}M_1\texttt{,} \ldots\texttt{,} M_n\texttt{)}$. A
constructor without parameter can be written $f\texttt{()}$ or simply
$f$. Constructors must be declared with $\texttt{fun
}f\texttt{/}n\texttt{.}$, as mentioned in the declarations. Names are
followed by their parameters between brackets:
$a\texttt{[}M_1\texttt{,} \ldots\texttt{,} M_n\texttt{]}$. A name
without parameter must be written $a\texttt{[]}$. Names are not declared before
being used. At first, constructors were designed to represent
cryptographic primitives, and names to represent fresh names created
by the protocol. However, there is no difference between names and
constructors from the point of view of the solver. We advise you to
use constructors rather than names, since the declaration of
constructors is a guarantee against typesetting errors.

So terms $M$ can be either constructor applications
$f\texttt{(}M_1\texttt{,} \ldots\texttt{,} M_n\texttt{)}$, name
applications $a\texttt{[}M_1\texttt{,} \ldots\texttt{,}\allowbreak
M_n\texttt{]}$, tuples $\texttt{(}M_1\texttt{,} \ldots\texttt{,}
M_n\texttt{)}$, or identifiers $x$ that can be used for variables or
constructors without parameters. Note that the term $(M)$ is different
from $M$: $(M)$ is a tuple with a single component containing $M$.

Facts can be the application of a predicate to terms
$p\texttt{:}M_1\texttt{,}\ldots\texttt{,}M_n$. (This is written this
way and not $p\texttt{(}M_1\texttt{,}\ldots\texttt{,}M_n\texttt{)}$
only for historical reasons.) They can also be $M \texttt{<>} M'$,
meaning $M$ is different from $M'$. Inequalities 
are allowed only in hypotheses of clauses, not in
conclusions, queries $\texttt{query }\nonterm{fact}\texttt{.}$, and
secrecy assumptions $\texttt{not }\nonterm{fact}\texttt{.}$.

Clauses can be $F_1 \texttt{ \& }\ldots\texttt{ \& }F_n\texttt{ -> }F$,
meaning $F_1$ and \ldots and $F_n$ implies $F$. They can also be
simply $F$, meaning that $F$ is true, without any hypothesis.

They can also be $F_1 \texttt{ \& }\ldots\texttt{ \& }F_n\texttt{ <-> }F$,
meaning $F_1$ and \ldots and $F_n$ is equivalent to $F$. This is allowed
only when $F_i$ do not contain inequality constraints, $\sigma F_i$ is
smaller than $\sigma F$ for all $\sigma$, and no two facts $F$ of
different equivalence declarations unify. It then generates the clauses
$F_1 \texttt{ \& }\ldots\texttt{ \& }F_n\texttt{ -> }F$, $F \texttt{ -> } F_i$,
and furthermore enables the replacement of $\sigma F$ with 
$\sigma F_1 \texttt{ \& }\ldots\texttt{ \& }\sigma F_n$ in all clauses.

The declaration $F_1 \texttt{ \& }\ldots\texttt{ \& }F_n\texttt{ <=> }F$
is a synonym for $F_1 \texttt{ \& }\ldots\texttt{ \& }F_n\texttt{ <-> }F$.
It is kept only for backward compatibility.


The goal of the system is to determine whether the facts declared in
queries can be derived from the given clauses.

\section{Input as process in extension of the pi calculus}

To give a pi calculus process as input to ProVerif, you have to
add the command line option \texttt{-in pi}, or to use
a filename that ends with \texttt{.pi}. You can then run ProVerif by:
\begin{quote}
$\texttt{./proverif -in pi }\nonterm{filename}$
\end{quote}
where $\nonterm{filename}$ references a file containing the process,
in the format explained below.

The keywords of this input system are {\tt among}, {\tt and}, {\tt can}, {\tt choice},
{\tt clauses}, {\tt data}, {\tt diff}, {\tt elimtrue},
{\tt else}, {\tt equation}, {\tt event}, {\tt fail}, {\tt
free}, {\tt fun}, {\tt if}, {\tt in}, {\tt let}, {\tt new}, {\tt
noninterf}, {\tt not}, {\tt nounif}, {\tt otherwise}, {\tt out}, {\tt param}, {\tt
phase}, {\tt putbegin}, {\tt pred}, {\tt private}, {\tt process}, {\tt
query}, {\tt reduc}, {\tt suchthat}, {\tt sync}, {\tt then}, {\tt weaksecret}, and {\tt where}.

\begin{figure}[tp]
\begin{align*}
&\nonterm{term} ::= \nonterm{ident}\texttt{(}\seq{term}\texttt{)}\\
&\phantom{\nonterm{term}}\mid\phantom{:=}\texttt{(}\seq{term}\texttt{)}\\
&\phantom{\nonterm{term}}\mid\phantom{:=}\nonterm{ident}\\
&\phantom{\nonterm{term}}\mid\phantom{:=}\texttt{choice[}\nonterm{term}\texttt{,}\nonterm{term}\texttt{]}\\
&\nonterm{pattern} ::= \nonterm{ident}\\
&\phantom{\nonterm{pattern}}\mid\phantom{:=}\texttt{(}\seq{pattern}\texttt{)}\\
&\phantom{\nonterm{pattern}}\mid\phantom{:=}\nonterm{ident}\texttt{(}\seq{pattern}\texttt{)}\\
&\phantom{\nonterm{pattern}}\mid\phantom{:=}\texttt{=}\nonterm{term}\\
&\nonterm{clause} ::= [( \nonterm{fact}
\texttt{ \& })^*  \nonterm{fact}\texttt{ ->}]\ \nonterm{fact}\ [\texttt{where }\seq{ident}\texttt{ can fail}]\\
&\phantom{\nonterm{clause}}\mid\phantom{:=} ( \nonterm{fact}
\texttt{ \& })^*  \nonterm{fact}\texttt{ <-> } \nonterm{fact}\ [\texttt{where }\seq{ident}\texttt{ can fail}]\\
&\phantom{\nonterm{clause}}\mid\phantom{:=} ( \nonterm{fact}
\texttt{ \& })^*  \nonterm{fact}\texttt{ <=> } \nonterm{fact}\ [\texttt{where }\seq{ident}\texttt{ can fail}]\\
&\nonterm{fact} ::=  \nonterm{ident} \texttt{:} \seq{term}\\
&\phantom{\nonterm{fact}}\mid\phantom{:=}\nonterm{term}\texttt{ <> }\nonterm{term}\\
&\phantom{\nonterm{fact}}\mid\phantom{:=}\nonterm{term}\texttt{ = }\nonterm{term}\\
&\nonterm{process} ::= \texttt{( }\nonterm{process}\texttt{ )}\\
&\phantom{\nonterm{process}}\mid\phantom{:=}\nonterm{ident}\\
&\phantom{\nonterm{process}}\mid\phantom{:=}\texttt{!~}\nonterm{process}\\
&\phantom{\nonterm{process}}\mid\phantom{:=}\texttt{0}\\
&\phantom{\nonterm{process}}\mid\phantom{:=}\texttt{new }\nonterm{ident}\texttt{; }\nonterm{process}\\
&\phantom{\nonterm{process}}\mid\phantom{:=}\texttt{if }\nonterm{fact}\texttt{ then }\nonterm{process}\ [\texttt{else }\nonterm{process}]\\
&\phantom{\nonterm{process}}\mid\phantom{:=}\texttt{in(}\nonterm{term}\texttt{, }\nonterm{pattern}\texttt{)}[\texttt{; }\nonterm{process}]\\
&\phantom{\nonterm{process}}\mid\phantom{:=}\texttt{out(}\nonterm{term}\texttt{, }\nonterm{term}\texttt{)}[\texttt{; }\nonterm{process}]\\
&\phantom{\nonterm{process}}\mid\phantom{:=}\texttt{let }\nonterm{pattern}\texttt{ = }\nonterm{term}\texttt{ in }\nonterm{process}\ [\texttt{else }\nonterm{process}]\\
&\phantom{\nonterm{process}}\mid\phantom{:=}\texttt{let }\seq{ident}\texttt{ suchthat }\nonterm{fact}\texttt{ in }\nonterm{process}\ [\texttt{else }\nonterm{process}]\\
&\phantom{\nonterm{process}}\mid\phantom{:=}\nonterm{process}\texttt{ | }\nonterm{process}\\
&\phantom{\nonterm{process}}\mid\phantom{:=}\texttt{event } \nonterm{term}\ [\texttt{; }\nonterm{process}]\\
&\phantom{\nonterm{process}}\mid\phantom{:=}\texttt{phase } n\ [\texttt{; }\nonterm{process}]\\
&\phantom{\nonterm{process}}\mid\phantom{:=}\texttt{sync } n\ [\texttt{; }\nonterm{process}]
\end{align*}
\caption{Grammar of processes}\label{fig:psyntax}
\end{figure}

\begin{figure}
\begin{align*}
&\nonterm{gterm} ::= \nonterm{ident}\texttt{(}\seq{gterm}\texttt{)}\\
&\phantom{\nonterm{gterm}}\mid\phantom{:=}\nonterm{ident}\texttt{[}\seq{gbinding}\texttt{]}\\
&\phantom{\nonterm{gterm}}\mid\phantom{:=}\texttt{(}\seq{gterm}\texttt{)}\\
&\phantom{\nonterm{gterm}}\mid\phantom{:=}\nonterm{ident}\\
&\nonterm{gbinding} ::= \texttt{!} n \texttt{ = } \nonterm{gterm}\\
&\phantom{\nonterm{gbinding}}\mid \phantom{:=}\nonterm{ident} \texttt{ = } \nonterm{gterm}\\
&\nonterm{gfact} ::=  \nonterm{ident} \texttt{:} \seq{gterm}\ [\texttt{phase }n]\\
&\phantom{\nonterm{gfact}}\mid\phantom{:=}\nonterm{gterm}\texttt{ <> }\nonterm{gterm}\\
&\phantom{\nonterm{gfact}}\mid\phantom{:=}\nonterm{gterm}\texttt{ = }\nonterm{gterm}\\
&\nonterm{realquery} ::= \nonterm{gfact} \texttt{ ==> } \nonterm{hyp}\\
&\nonterm{hyp} ::= \nonterm{hyp} \texttt{ | } \nonterm{hyp}\\
&\phantom{\nonterm{hyp}}\mid\phantom{:=}\nonterm{hyp} \texttt{ \&\ } \nonterm{hyp}\\
&\phantom{\nonterm{hyp}}\mid\phantom{:=}\nonterm{gfact}\\
&\phantom{\nonterm{hyp}}\mid\phantom{:=}\texttt{(} \nonterm{hyp}\texttt{)}\\
&\phantom{\nonterm{hyp}}\mid\phantom{:=}\texttt{(} \nonterm{realquery}\texttt{)}\\
&\nonterm{query} ::= \texttt{putbegin ev:}\seq{ident}[\texttt{; }\nonterm{query}]\\
&\phantom{\nonterm{query}}\mid\phantom{:=}\texttt{putbegin evinj:}\seq{ident}[\texttt{; }\nonterm{query}]\\
&\phantom{\nonterm{query}}\mid\phantom{:=}\texttt{let }\nonterm{ident}\ \texttt{=}\ \nonterm{gterm}[\texttt{; }\nonterm{query}]\\
&\phantom{\nonterm{query}}\mid\phantom{:=}\nonterm{gfact}[\texttt{; }\nonterm{query}]\\
&\phantom{\nonterm{query}}\mid\phantom{:=}\nonterm{realquery}[\texttt{; }\nonterm{query}]\\[2mm]
&\nonterm{gtermformat} ::= \textit{ same as }\nonterm{gterm} \textit{ with additional } \texttt{*}\nonterm{ident}\\
&\nonterm{gfactformat} ::= \nonterm{ident} \texttt{:} \seq{gtermformat}\ [\texttt{phase }n]
\end{align*}
\caption{Grammar of queries and \texttt{nounif}}\label{fig:psyntaxq}
\end{figure}

The input file consists of a list of declarations, followed by the
keyword {\tt process} and a process:
\[\nonterm{declaration}^*\ {\tt process}\ \nonterm{process}\]
The syntax of terms and processes is given in Figures~\ref{fig:psyntax}
and~\ref{fig:psyntaxq},
using the same conventions as in Section~\ref{sect:horninput}.

Declarations can be any of the following:
\begin{itemize}

\item $\texttt{param }\nonterm{name}\texttt{ = }\nonterm{value}\texttt{.}$

This declaration sets the value of configuration parameters.  The
cases mentioned in Section~\ref{sect:horninput} are supported, as well
as the following ones:

\begin{itemize}

\item $\texttt{param attacker = active.}$\\
$\texttt{param attacker = passive.}$

Indicates whether the attacker is active (\texttt{param attacker =
active.}) or passive (\texttt{param attacker = passive.}). An active
attacker can read messages, compute, and send messages. A passive
attacker can read messages and compute but not send messages.

\item $\texttt{param keyCompromise = none.}$\\
$\texttt{param keyCompromise = approx.}$\\
$\texttt{param keyCompromise = strict.}$

By default (\texttt{param keyCompromise = none.}), it is assumed that
session keys are not a priori compromised. Otherwise, it is assumed
that some session keys are compromised (known by the adversary). Then
the system determines whether the secrets of other sessions can be
obtained by the adversary. In this case, the names that occur in
queries always refer to names of non-compromised sessions (the
attacker has all names of compromised sessions), and the events that
occur before an arrow \texttt{==>} in a query are executed only in
non-compromised sessions. 
% The modeling with \texttt{param
%   keyCompromise = approx.} is more approximate than with \texttt{param
%   keyCompromise = strict.}, that is, the chances of finding a false
% attack are greater with \texttt{param keyCompromise = approx.}.
With \texttt{param keyCompromise = approx.}, the compromised sessions
are considered as executing possibly in parallel with non-compromised ones.
With \texttt{param keyCompromise = strict.}, the compromised sessions
are finished before the non-compromised ones begin. The chances of finding an
attack are greater with \texttt{param keyCompromise = approx.}. (It may
be a false attack due to the approximations made in the verifier.)

\item $\texttt{param movenew = false.}$\\
$\texttt{param movenew = true.}$

Sets whether the system should try to move restrictions under inputs,
to have a more precise analysis (\texttt{param movenew = true.}), or
leave them where the user has put them (\texttt{param movenew =
false.}).

\item $\texttt{param predicatesImplementable = check.}$\\
$\texttt{param predicatesImplementable = nocheck.}$

Sets whether the system should check that predicate calls are
implementable. See the \texttt{clauses} declaration below for more
details on this check. It is advised to leave the check turned on,
as it is by default. Otherwise, the semantics of the processes
may not be well-defined.

\item $\texttt{param verboseClauses = none.}$\\
$\texttt{param verboseClauses = explained.}$\\
$\texttt{param verboseClauses = short.}$

When \texttt{verboseClauses = none}, ProVerif does not display
the clauses it generates. 
When \texttt{verboseClauses = short}, it displays them.
When \texttt{verboseClauses = explained}, it adds an
English sentence after each clause it generates to explain where this
clause comes from.

\item $\texttt{param explainDerivation = true.}$\\
$\texttt{param explainDerivation = false.}$

When \texttt{explainDerivation = true}, ProVerif explains in English
each step of the derivation (returned in case of failure of a proof).
This explanation refers to program points in the given process.
When \texttt{explainDerivation = false}, it displays the derivation
by referring to the clauses generated initially.

\item $\texttt{param removeUselessClausesBeforeDisplay = true.}$\\
$\texttt{param removeUselessClausesBeforeDisplay = false.}$

When \texttt{removeUselessClausesBeforeDisplay = true}, ProVerif removes
subsumed clauses and tautologies from the initial clauses before
displaying them, to avoid showing many useless clauses.
When \texttt{removeUselessClausesBeforeDisplay = false}, all generated
clauses are displayed.

\item $\texttt{param reconstructTrace = true.}$\\
$\texttt{param reconstructTrace = false.}$

With $\texttt{param reconstructTrace = true.}$, when a query cannot be
proved, the tool tries to build a pi calculus execution trace that is
a counter-example to the query~\cite{Allamigeon05}. 

This feature is currently incompatible with key compromise
($\texttt{param keyCompromise = approx.}$ or $\texttt{param
  keyCompromise = strict.}$).

Moreover, for \texttt{noninterf} and \texttt{choice}, it reconstructs
a trace, but this trace may not always prove that the property is
wrong: for \texttt{noninterf}, it reconstructs a trace until a program
point at which the process behaves differently depending on the value
of the secret (takes a different branch of a test, for instance), but
this different behavior is not always observable by the adversary;
similarly, for \texttt{choice}, it reconstructs a trace until a
program point at which the process using the first argument of
\texttt{choice} behaves differently from the process using the second
argument of \texttt{choice}.

For injective queries, the trace reconstruction proceeds in two steps.
In the first step, it reconstructs a trace that corresponds to the
derivation found by resolution. This trace generally executes events
once, so does not contradict injectivity. In a second step, it tries
to reconstruct a trace that executes certain events twice while it executes
other events once, in such a way that injectivity is really contradicted.
This second step may fail even when the first one succeeds.
For non-injective queries (including secrecy), when a trace is found,
it is a counter-example to the query, which is then false.

\item $\texttt{param traceBacktracking = true.}$\\
$\texttt{param traceBacktracking = false.}$

Allow or disable backtracking when reconstructing traces.
In most cases, when traces can be found, they are found without
backtracking. Disabling backtracking makes it possible to display the
trace during its computation, and to forget previous states of the trace.
This reduces memory consumption, which can be necessary for
reconstructing very big traces.

\item \texttt{param unifyDerivation = true.}\\
\texttt{param unifyDerivation = false.}

When set to \texttt{true}, activates a heuristic that increases the 
chances of finding a trace that corresponds to a derivation. 
This heuristic unifies messages received by the same input
(same ocurrence and same session identifiers) in the derivation.
Indeed, these messages must be equal if the derivation corresponds to
a trace.

\item $\texttt{param traceDisplay = short.}$\\
$\texttt{param traceDisplay = long.}$\\
$\texttt{param traceDisplay = none.}$

Choose the format in which the trace is displayed after trace reconstruction.
By default ($\texttt{param traceDisplay = short.}$), outputs the labels of
a labeled reduction.
With \texttt{param traceDisplay = long.}, outputs the current state
before each input and before and after each I/O reduction, as well as 
the list of all executed reductions.
With \texttt{param traceDisplay = none.}, the trace is not displayed.


\end{itemize}

\item $[\texttt{private}]\ \texttt{fun }\nonterm{ident}\texttt{/}n\texttt{.}$

$\texttt{fun }f\texttt{/}n\texttt{.}$ declares a function symbol $f$
of arity $n$. This function symbol is a constructor. When
$\texttt{private}$ is not present, the function can be applied by the
attacker. When $\texttt{private}$ is present, the function cannot be
applied by the attacker. This last case is useful to model tables of
keys stored in a server, for instance. Only the server can use the
table to get associations between host names and keys.

\item $\texttt{data }\nonterm{ident}\texttt{/}n\texttt{.}$

$\texttt{data }f\texttt{/}n\texttt{.}$ declares a data function symbol
$f$ of arity $n$.  Data function symbols are similar to tuples: the
adversary can construct and decompose them.


\item $[\texttt{private}]\ \texttt{reduc }(\nonterm{ident}\texttt{(}\seq{term}\texttt{) = }\nonterm{term}\texttt{;})^*$\\
$\phantom{[\texttt{private}]\ \texttt{reduc }}\nonterm{ident}\texttt{(}\seq{term}\texttt{) = }\nonterm{term}\texttt{.}$

This declares destructors:
\begin{align*}
\texttt{reduc }&f\texttt{(}M_1\texttt{,} \ldots\texttt{,} M_n\texttt{) = }M_0\texttt{;}\\
&f\texttt{(}M'_1\texttt{,} \ldots\texttt{,} M'_n\texttt{) = }M'_0\texttt{;}\\
&\ldots\\
&f\texttt{(}M''_1\texttt{,} \ldots\texttt{,} M''_n\texttt{) = }M''_0\texttt{.}
\end{align*}
declares the destructor $f$, of arity $n$, with the given rewrite
rules.  When a term $f(M_1, \ldots, M_n)$ is met, it is replaced by
$M_0$, and similarly for the other rules. When no rule can be
applied, the destructor is not defined; the process blocks. 
The destructor must be deterministic, that is, when several rules can be
applied, they must all yield the same result.
Otherwise, an error message is displayed.
The terms
$M_0, \ldots, M_n, M'_0, \ldots, M'_n,\ldots$ must contain only
variables and constructors.

\item $[\texttt{private}]\ \texttt{reduc }(\nonterm{ident}\texttt{(}\seq{mfterm}\texttt{) = }\nonterm{mfterm}\ [\texttt{where }\seq{ident}\texttt{ can fail}]\texttt{ otherwise })^*$\\
$\phantom{[\texttt{private}]\ \texttt{reduc }(}\nonterm{ident}\texttt{(}\seq{mfterm}\texttt{) = }\nonterm{mfterm}\ [\texttt{where }\seq{ident}\texttt{ can fail}]\texttt{.}$

This is an extended declaration for destructors.
The terms $\nonterm{nfterm}$ can be either the special constant 
\texttt{fail}, which represents the failure of a term, or terms $\nonterm{term}$
containing only variables and constructors.

\begin{align*}
\texttt{reduc }&f\texttt{(}M_1\texttt{,} \ldots\texttt{,} M_n\texttt{) = }M_0\texttt{ where }x_1, \ldots, x_k\texttt{ can fail}\\
&\texttt{otherwise }f\texttt{(}M'_1\texttt{,} \ldots\texttt{,} M'_n\texttt{) = }M'_0\texttt{ where }x'_1, \ldots, x'_{k'}\texttt{ can fail}\\
&\texttt{otherwise }\ldots\\
&\texttt{otherwise }f\texttt{(}M''_1\texttt{,} \ldots\texttt{,} M''_n\texttt{) = }M''_0\texttt{ where }x''_1, \ldots, x''_{k''}\texttt{ can fail.}
\end{align*}
declares the destructor $f$ with the given rewrite rules.
When a term $f(N_1, \ldots, N_n)$ is met, if it is an instance of
$f(M_1, \ldots, M_n)$, then it is replaced with the corresponding instance of $M_0$.
Otherwise, if it is an instance of $f(M'_1, \ldots, M'_n)$, then it is replaced with the corresponding instance of $M'_0$, and so on.
Each rewrite rule is applied only if the previous rewrite rules cannot be applied. Hence, the destructor is deterministic by construction.

The variables $x_1, \ldots, x_k$ mentioned in ``\texttt{where }$x_1, \ldots, x_k$\texttt{ can fail}'' can be replaced
with any term, and also with the constant \texttt{fail}. Such variables are allowed only at the root of terms $M_1, \ldots, M_n, M_0$.
When there is no such variable, the indication ``\texttt{where }$x_1, \ldots, x_k$\texttt{ can fail}'' is omitted.
Other variables can be replaced with any term but not with the constant \texttt{fail}; they can occur anywhere.
The constant \texttt{fail} used in $N_i (i>0)$ represents that the $i$-th argument of $f$ failed; when it is used in the result $M_0$, it means that the evaluation of $f$ itself fails in this case. Using this constant \texttt{fail}, we can, for instance, define destructors that do not fail even if some of their arguments fail.

Examples include:
\begin{verbatim}
fun true/0.
fun false/0.
reduc equal(x,x) = true
      otherwise equal(x,y) = false.
\end{verbatim}
defines an equality destructor, which returns true when its two arguments
are equal and false otherwise. This destructor fails when one of its arguments
fails.

\begin{verbatim}
reduc test(true, x, y) = x where x, y can fail
      otherwise test(false, x, y) = y where x,y can fail.
\end{verbatim}
defines a test destructor that returns its second argument when
its first argument is true, its third argument when its first argument
is false, and fails otherwise. Notice that \texttt{x} and \texttt{y}
are allowed to be \texttt{fail}, so that, when the first argument
is true, the second argument is returned even if the third argument fails,
and symmetrically in the other case.

\item $\texttt{equation }\nonterm{term}\texttt{ = }\nonterm{term}\texttt{.}$

$\texttt{equation }M_1\texttt{ = }M_2$ says that the terms $M_1$ and
$M_2$ are in fact equal. The function symbols in the equation should
be only already declared constructors. The treatment of equations is
still rather limited. The equation $f(x,g(y)) =
f(y,g(x))$, used for Diffie-Hellman key agreements, is known to
work. The system may not terminate when more complex equations are
entered. 

\item $\texttt{pred }\nonterm{ident}\texttt{/}n\ \seq{ident}\texttt{.}$

$\texttt{pred }p\texttt{/}n\ i_1, \ldots, i_n\texttt{.}$ declares a new
predicate $p$, of arity $n$, with special properties described by
$i_1, \ldots, i_n$. Currently, the allowed elements for $i_1,
\ldots, i_n$ are \texttt{block} and 
\texttt{memberOptim}. See the section on the Horn clause input for
more details on the effect of these properties. 

All predicates must be declared by such a declaration before being
used. The predicates $\texttt{attacker}$, $\texttt{mess}$, $\texttt{ev}$,
and $\texttt{evinj}$ are reserved and cannot be declared.

\item $\texttt{query }\nonterm{query}\texttt{.}$

This declaration tells the system which properties we want to prove.
Its syntax is given in Figure~\ref{fig:psyntaxq}.
\begin{itemize}

\item A term $M$ is a query can contain as usual constructor applications and
variables, but also constructs $a[\ldots]$ or simply $a$ that designate names
created by the restriction $\texttt{new }a$. $a[]$ designates any name
created by the restriction $\texttt{new }a$. $a[v_1 = M_1, \ldots, v_n
= M_n]$ designates any name created by the restriction $\texttt{new
}a$ when the variables $v_1, \ldots, v_n$ have value $M_1, \ldots,
M_n$ respectively. These variables must be in scope at the considered
restriction. A special variable $\texttt{!}n$ corresponds to the 
session identifier of the $n$-th replication, starting from the
top of the process with $n=1$ for the first replication. Session
identifiers should only be bound to variables. By using the same
variable several times, one can express that two names should be 
created in the same copy of the process.

Note that, to avoid ambiguities, several
restrictions in the processes should not create the same name. (When
several restrictions create the same name and one tries to refer
to this name in a query, an error message is displayed.)

Also note that one should be careful of not mixing names and variables:
An identifier that is defined by a restriction represents any name
defined at that restriction (possibly different names for different
occurrences), a free name of the process represents that name, 
while other identifiers are variables that represent any
term (the same term for all occurrences of the same variable).

A term $M$ in a query must not contain destructors. 

\item Facts in queries can be $p:M_1, \ldots, M_n\ [\texttt{phase }n]$,
with the following meanings:

\begin{itemize}

\item $\texttt{attacker:}M$ means that the attacker may have $M$
in some phase ($M$ is not secret).

\item $\texttt{attacker:}M\texttt{ phase }n$ means that the attacker
may have $M$ in phase $n$.

\item $\texttt{mess:}M,N$ means that the message $N$ may be sent
on channel $M$ in the last phase.

\item $\texttt{mess:}M,N\texttt{ phase }n$ means that the message $N$
may be sent on channel $M$ in phase $n$.

\item $\texttt{ev:}f(M_1, \ldots, M_n)$ means that the event
$\texttt{event }f(M_1, \ldots, M_n)$ may be executed. There must exist
some $\texttt{event }f(M'_1, \ldots, M'_n)$ instruction with the same
function symbol $f$ in the process.

\item $\texttt{evinj:}f(M_1, \ldots, M_n)$ means that the event
$\texttt{event }f(M_1, \ldots, M_n)$ may be executed,
and that furthermore we want to prove injective correspondences
for this event (see examples below).

\item $p:M_1, \ldots, M_n$, where $p$ is a user-defined predicate (see
the \texttt{pred} and \texttt{clauses} declarations), means that the
corresponding fact is true.

\end{itemize}
Note that the \texttt{phase} indication is not allowed for
\texttt{ev}, \texttt{evinj}, and user-defined predicates. One can
also use the facts $M\texttt{ = }N$ and $M\texttt{ <> }N$.

\item The elementary query $\nonterm{realquery}$ can be 
$F \texttt{ ==> }\phi$ where $\phi$ is formed from conjunctions
and disjunctions of facts.
The query is true if and only if, when $F$ is true, then $\phi$ is true.
The query can also be a fact $F$ (see $\nonterm{query}$):
the answer can be that $\texttt{not }F$ is true, i.e. the fact $F$ is false,
or that $\texttt{not }F$ cannot be proved, when the system
finds a (possibly false) attack that would make $F$ true. (This query is a
shorthand for $F \texttt{ ==> false}$.) Here are some examples:
\begin{itemize}

\item $\texttt{query attacker:}M$ determines whether the attacker
may have $M$. $\texttt{not attacker:}M$ is true when $M$ is secret.

\item $\texttt{query ev:}f(M_1, \ldots, M_n)$ determines whether the event
$\texttt{event }f(M_1, \ldots, M_n)$ may be executed.
$\texttt{not ev:}f(M_1, \ldots, M_n)$ is true when the event
$\texttt{event }f(M_1, \ldots, M_n)$ can never be executed.

\item $\texttt{query ev:}f(x_1, \ldots, x_n)\texttt{ ==> ev:}f'(x_1,
\ldots, x_n)$ is non-injective agreement: it is true when, if the
event $f(x_1, \ldots, x_n)$ has been executed, then the event $f'(x_1,
\ldots, x_n)$ must have been executed (before the event $f(x_1,
\ldots, x_n)$).

\item $\texttt{query evinj:}f(x_1, \ldots, x_n)\texttt{ ==>
evinj:}f'(x_1, \ldots, x_n)$ is injective agreement: it is true when,
for each executed event $f(x_1, \ldots, x_n)$, there exists a distinct
executed event $f'(x_1, \allowbreak \ldots, \allowbreak 
x_n)$ (and $f'(x_1, \ldots, x_n)$ is
executed before $f(x_1, \ldots, x_n)$).

\item $\texttt{query evinj:}f(M_1)\texttt{ ==> evinj:}f'(M_2)\texttt{
\& ev:}f''(M_3)$ is true if and only if for each executed event
$f(M_1)$ there exists a distinct executed event $f'(M_2)$ and an
executed event $f''(M_3)$. (The event $f''(M_3)$ can be the
same for several different events $f(M_1)$, since it is marked with
\texttt{ev} and not \texttt{evinj}.)

\item $\texttt{query evinj:}f(M_1)\texttt{ ==> evinj:}f'(M_2)\texttt{
| ev:}f''(M_3)$ is true if and only if for each executed event
$f(M_1)$ either there exists a distinct executed event $f'(M_2)$ or an
event $f''(M_3)$ has been executed. 

\end{itemize}
Note that using \texttt{evinj} or \texttt{ev} before the arrow
\texttt{ ==> } does not change the meaning of the query. It is
important only after the arrow.
%except when using putbegin!!!

We can also use \emph{nested queries}: queries in which some of
the events after the arrow \texttt{==>} are replaced with queries.
For instance, $F \texttt{ ==> (}F' \texttt{ ==> }F''\texttt{)}$.
For this query to be interesting, $F'$ and $F''$ must be events.
This query is true if and only if, when $F$ is true, the event $F'$ is 
executed, and $F''$ is executed \emph{before} $F'$. (In contrast,
the query $F \texttt{ ==> }F' \texttt{ \& }F''$ would not order $F'$
and $F''$.) If $F$ is also an event, $F'$ is executed before $F$.

We can use more complex queries in this style, such as 
\[F_0 \texttt{ ==> (}F_1 \texttt{ ==> (}F_2 \texttt{ ==> (}
F_3 \texttt{ ==> }F_4\texttt{)))}\]
which is true if and only if, when $F_0$ is true,
$F_4, F_3, F_2, F_1$ have been executed in that order, or
\[F_0 \texttt{ ==> (}F_1 \texttt{ ==> }F_2 \texttt{) \& (}
F_3 \texttt{ ==> }F_4\texttt{)}\]
which is true if and only if, when $F_0$ is true,
$F_2$ has been executed before $F_1$ and $F_4$ before $F_3$.

\item The full query $\nonterm{query}$ consists of a list containing
as elements queries of the form $\nonterm{realquery}$ or $\nonterm{gfact}$
as described above, as well as the following two elements:
\begin{itemize}

\item $\texttt{putbegin ev:}f_1, \ldots, f_n$ or $\texttt{putbegin
evinj:}f_1, \ldots, f_n$ instructs the system to consider active
``begin'' events the events $f_1(\ldots)$, \ldots, $f_n(\ldots)$.
This means that when such an event needs to be executed to trigger
another action, a begin fact is going to appear in the hypothesis
of the corresponding clause. This is useful when the exact events
that should appear in a query are unknown. For instance,
with the query $\texttt{query putbegin ev:}f\texttt{; ev:}f'(x)$,
the system generates clauses that conclude $\textsf{end:}f'(M)$,
and by manual inspection of the facts $\textsf{begin:}f(M')$
that occur in their hypothesis, one can infer the full query:
\[\texttt{query ev:}f'(\ldots) \texttt{ ==> ev:}f(\ldots).\]

When using \texttt{evinj:}, the activated begin events contain
an environment that can be used to prove injective correspondences.  

(This way of writing queries simulates what happened in older
versions of ProVerif, up to version 1.09.)

\item $\texttt{let }x = M$ binds the variable $x$ to the term $M$.
This is especially useful to designate several times the same name.
For example, $\texttt{query let x = a[]; attacker:f(x,x)}$ determines
whether the attacker may have $f(x,x)$ where $x$ is any name created
by a restriction $\texttt{new a}$. In contrast, $\texttt{query
attacker:f(a[],a[])}$ determines whether the attacker may have
$f(x,y)$ where $x$ and $y$ are (possibly different) names created by a
restriction $\texttt{new a}$. A variable bound by $\texttt{let }x = M$ must
be bound before being used in the following of the query.

\end{itemize}
All queries of the list included in a single \texttt{query} declaration
are evaluated by building one set of clauses and performing resolution
on it, while different \texttt{query} declarations are evaluated by
rebuilding a new set of clauses from scratch. So the way queries are
grouped influences the sharing of work between different queries, so
the speed of the system. The main idea is that one should group
queries that involve the same events, but separate queries that
involve different events, because the more events appear in the query,
the more complex the generated clauses are, which can slow down the
system considerably, especially on complex examples. If one does not
want to optimize, one can simply put a single query in each
\texttt{query} declaration.

\end{itemize}

\item $\texttt{noninterf }\seq{interfspec}\texttt{.}$\\
where $\nonterm{interfspec} ::= \nonterm{ident}\ [\texttt{among (}\seq{term}\texttt{)})]$

$\texttt{noninterf }n_1\ \texttt{among}\ (S_1), \ldots, n_k\ \texttt{among}\ (S_k)\texttt{.}$ tells the system to
prove strong secrecy for the secrets $n_1, \ldots, n_k$. That is, the
system shows that several versions of the given process that
differ by their values of $n_1, \ldots, n_k$ are bisimilar
(therefore they are testing equivalent, observationally equivalent, \ldots -- 
see~\cite{Blanchet04} for more details).
When the $\texttt{among}\ (S_i)$ indication is present, it means that
$n_i$ can take its values only inside $S_i$. When it is absent,
$n_i$ can take any value not containing bound names (or private free names).

Note that the $\texttt{let }\ldots \texttt{ suchthat}$ construct
is incompatible with the test of strong secrecy. What the solver does
in this case cannot be done when the input is given under
the form of Horn clauses (because the simplifications done by the system
are sound only for particular clauses that correspond to those 
generated from a process; they do not make sense for general clauses).

\item $\texttt{weaksecret }\nonterm{ident}\texttt{.}$

$\texttt{weaksecret }n$ tells the system to check whether an attacker
guessing the value of $n$ can verify its guess offline. This is useful
when $n$ is a weak secret, such as a password, that an attacker could
guess by exhaustive enumeration.

\item $\texttt{nounif }\nonterm{gfactformat}[/n]\ [b]\texttt{.}$ where $b = \texttt{;}\nonterm{ident} \texttt{ = }\nonterm{gtermformat} \texttt{;}\ldots\texttt{;}\nonterm{ident} \texttt{ = }\nonterm{gtermformat}$

  $\texttt{nounif }F[/n]$ modifies the selection of facts to be
  resolved upon, to avoid resolving facts that match $F$. A fact
  $F'$ matches $F$ if and only if $F' = \sigma F$ for some substitution 
  $\sigma$ that maps variables always marked with a star \texttt{*} to
  any term and variables that occur at least once without star to a variable.
  The optional integer $n$ indicates how much we should
  avoid resolution on facts that match $F$: the greater $n$, the more
  such resolutions will be avoided.

  The only supported facts $F$ are $\texttt{attacker:}M\
  [\texttt{phase }n]$, $\texttt{mess:}M,N\ [\texttt{phase }n]$, and
  $p:M_1, \ldots, M_n$ when $p$ is a user-defined predicate.

The fact $F$ can contain as usual constructor applications and
variables, but also constructs $a[\ldots]$, as in queries.
(See the declaration \texttt{query} above.)

$\texttt{nounif }F\ [/n]\texttt{;}x_1 = M_1\texttt{;} \ldots
\texttt{;}x_n = M_n$ corresponds to $\texttt{nounif }F\{M_n / x_n\}
\ldots \{ M_1/x_1 \}$.  ($x_j$ may occur in $M_i$ only when $i>j$).
This construct is
especially useful to designate several times the same name: for
example $\texttt{nounif attacker:f(x,x);x = a[]}$ prevents resolution
on $\mathtt{attacker:f(a[M_1, \ldots, M_m],a[M_1, \ldots, M_m])}$
while $\texttt{nounif attacker:f(a[],a[])}$ prevents resolution
on $\mathtt{attacker:f(a[M_1, \ldots, M_m],a[M'_1, \ldots, M'_m])}$.

In $\nonterm{gfactformat}$ and $\nonterm{gtermformat}$, an identifier
without arguments $a$ stands for $a()$ when the function $a$ has been
defined (it must then have arity 0), for $a[]$ when a restriction
$\texttt{new }a$ occurs in the process, and for a variable otherwise.

\item $\texttt{elimtrue }\nonterm{fact}\ [\texttt{where }\seq{ident}\texttt{ can fail}]\texttt{.}$:
same as for the Horn clause front-end (to be used with user-declared
predicates).

In addition, the fact is allowed to contain destructors, and 
the indication ``\texttt{where }$x_1, \ldots, x_n$\texttt{ can fail}''
specifies that the variables $x_1, \ldots, x_n$ can take the special
value \texttt{fail}, which represents the failure of the evaluation
of a destructor. When it is omitted, no variable can take the value 
\texttt{fail}. (See the definition of destructors by \texttt{reduc} 
\ldots \texttt{otherwise} for more details.)

\item $\texttt{not }\nonterm{gfact}\ [b]\texttt{.}$ where $b = \texttt{;}\nonterm{ident} \texttt{ = }\nonterm{gterm} \texttt{;}\ldots\texttt{;}\nonterm{ident} \texttt{ = }\nonterm{gterm}$

$\texttt{not }F$ adds the assumption that $F$ is not derivable (or all
instances of $F$ when $F$ contains variables). This speeds up the
solving process. At the end of the solving process, the system checks
that $F$ is indeed not derivable. If it is not, the proof fails with
an error message. The only supported facts $F$ are
$\texttt{attacker:}M\ [\texttt{phase }n]$, $\texttt{mess:}M,N\
[\texttt{phase }n]$, and $p:M_1, \ldots, M_n$ when $p$ is a
user-defined predicate. 

For $\texttt{not attacker:}M\ [\texttt{phase }n]$, when $\texttt{phase
}n$ is present, it means that $M$ is secret in phases up to phase
$n$. When $\texttt{phase }n$ is absent, it means that $M$ is secret in
all phases.  For backward compatibility, the syntax $\texttt{not
}M\ [\texttt{phase }n]\ [b]\texttt{.}$ is also supported
as an abbreviation of $\texttt{not attacker:}M\
[\texttt{phase }n]\ [b]\texttt{.}$.

The fact $F$ can contain as usual constructor applications and
variables, but also constructs $a[\ldots]$, as in queries.
(See the declaration \texttt{query} above.)

$\texttt{not }F\texttt{;}x_1 = M_1\texttt{;} \ldots \texttt{;}x_n =
M_n$ corresponds to $\texttt{not }F\{M_n / x_n\} \ldots \{ M_1/x_1
\}$.  ($x_j$ may occur in $M_i$ only when $i>j$). This construct is
especially useful to designate several times the same name: for
example $\texttt{not attacker:f(x,x);x = a[]}$ means that $\texttt{f(x,x)}$ is
secret when $\texttt{x}$ is any name created by $\texttt{new a}$,
while $\texttt{not attacker:f(a[],a[])}$ means that $\texttt{f(x,y)}$ is secret
when $\texttt{x}$ and $\texttt{y}$ are any (possible different) names
created by $\texttt{new a}$.

In $\nonterm{gfact}$, an identifier without arguments $a$
stands for $a()$ when the function $a$ has been defined (it must then
have arity 0), for $a[]$ when a restriction $\texttt{new }a$ occurs
in the process, and for a variable otherwise.

\item $[\texttt{private}]\ \texttt{free }\seq{ident}\texttt{.}$

$\texttt{free }i_1, \ldots, i_n\texttt{.}$ declares the free names
$i_1, \ldots, i_n$. When the keyword \texttt{private} is present, the
name is not known by the adversary, whereas by default, it is known by the 
adversary. When a name occurs free in the process, and is not
declared by such a declaration, a warning is displayed.  We strongly
encourage you to declare all free names of your processes.  Indeed, an
unexpected free name corresponds in general to a typesetting error,
and the warning might become an error in a future release.

\item $\texttt{clauses}\ (\nonterm{clause}\texttt{;})^*\ 
\nonterm{clause}\texttt{.}$

This introduces clauses that define predicates. These predicates can
be used in $\texttt{let}\ \ldots\ \texttt{ suchthat}$
processes and in tests $\texttt{if}\ \ldots\ \texttt{then}$.  
Note that there is an implementability
condition. Essentially, for each predicate invocation, we bind
variables in the conclusion of the clauses that define this predicate
and whose position corresponds to bound arguments of the predicate
invocation. Then, when evaluating hypotheses of clauses from left to
right, all variables of predicates must get bound by the corresponding
predicate call. Recursive definitions of predicates are allowed.

The meaning of clauses is the same as for the Horn clauses input system.
In addition, the clauses are allowed to contain destructors, and 
the indication ``\texttt{where }$x_1, \ldots, x_n$\texttt{ can fail}''
appended to clauses
specifies that the variables $x_1, \ldots, x_n$ can take the special
value \texttt{fail}, which represents the failure of the evaluation
of a destructor. When it is omitted, no variable can take the value 
\texttt{fail}. (See the definition of destructors by \texttt{reduc} 
\ldots \texttt{otherwise} for more details.)
The clauses apply only when the arguments of all facts in them do not fail: 
for facts $p(M_1, \ldots, M_k)$, $M_1, \ldots, M_k$ do not fail, 
for equalities $M_1 = M_2$ and inequalities $M_1 <> M_2$,
$M_1$ and $M_2$ do not fail.

\item $\texttt{let }\nonterm{ident}\texttt{ = }\nonterm{process}\texttt{.}$

Defines $\nonterm{ident}$ as the process
$\nonterm{process}$. $\nonterm{ident}$ can be used inside the
definition of processes. If the process contains free names or
variables, they can be bound when $\nonterm{ident}$ is used. (So, this
is a kind of macro-expansion rather than a real definition.)

\end{itemize}

In the syntax of processes,
\begin{itemize}

\item The pattern $\nonterm{ident}$ matches any term, and binds the
given variable identifier to the matched term. The pattern
$(\seq{pattern})$ matches tuples (and each component of the tuple is
recursively matched by the given patterns). The pattern
$f(\seq{pattern})$ matches terms of the form $f(M_1, \ldots, M_n)$
and the subterms $M_i$ are recursively matched by the given patterns,
where $f$ is a data function symbol (see the $\texttt{data}$ declaration).
When $f$ is not a data function symbol, such a construction is not
allowed. The pattern
$\texttt{=}\nonterm{term}$ matches a term that is equal to the given
$\nonterm{term}$. (This is equivalent to an equality test.)

\item Parentheses are just used to clarify associativity of parallel
compositions, and which processes are replicated.

\item An identifier $x$ must be defined by a previous declaration
$\texttt{let }x\texttt{ = }P$. It is then equivalent to having a copy
of $P$ instead of $x$.

\item The replication $\texttt{!~}P$ executes an unbounded number of
copies of $P$ in parallel: $P \mid P \mid P \mid \ldots$.

\item The nil process $\texttt{0}$ does nothing.

\item The restriction $\texttt{new }a\texttt{;}P$ creates a new name
$a$, then executes $P$.

\item The test $\texttt{if }f\texttt{ then }P\texttt{ else }Q$
executes $P$ when the fact is true. Otherwise, it executes $Q$. The
process $\texttt{if }f\texttt{ then }P$ is equivalent to $\texttt{if
}f\texttt{ then }P\texttt{ else 0}$. Note that the predicate calls are
subject to an implementability condition (see the \texttt{clauses}
declaration above). Equality and inequality tests are always
implementable.

\item The input $\texttt{in(}c\texttt{,}p\texttt{);}P$ inputs a
message on channel $c$, and executes $P$ after matching the input
message with $p$, and binding the variables contained in $p$. When a
message does not match $p$, it cannot be input by this construct. The
channel $c$ can be any term. The process
$\texttt{in(}c\texttt{,}x\texttt{)}$ is equivalent to
$\texttt{in(}c\texttt{,}x\texttt{);0}$.

\item The output $\texttt{out(}c\texttt{,}M\texttt{);}P$ outputs the
message $M$ on the channel $c$, then executes $P$. ($c$ can be any
term.) The process $\texttt{out(}c\texttt{,}M\texttt{)}$
is equivalent to $\texttt{out(}c\texttt{,}M\texttt{);0}$.

\item The let binding $\texttt{let }p\texttt{ = }M\texttt{ in }P\ [\texttt{else }Q]$
executes $P$ after matching the term $M$ with the pattern $p$, and 
binding the variables contained in $p$. If the term $M$ does not match
the pattern $p$, the process blocks, or executes $Q$ when the 
\texttt{else} clause is present.

\item The binding $\texttt{let }x_1, \ldots, x_n\texttt{ such that
}f\texttt{ in }P\ [\texttt{else }Q]$ binds new variables $x_1, \ldots,
x_n$, such that $f$ is true, then executes $P$. If such a binding is
impossible, it executes $Q$. Note that the predicate calls are subject
to an implementability condition (see the \texttt{clauses} declaration
above). Facts $M\texttt{ <> }N$ are not allowed in $f$, because of
this implementability condition (they make sense only when all
variables are already bound; in this case, using the \texttt{else}
clause of a \texttt{if} is more appropriate).

\item The parallel composition $P_1 \texttt{|} P_2$ executes $P_1$ and
$P_2$ in parallel.

\item The event $\texttt{event } M\texttt{;}P$ emits the event
$\texttt{event}(M)$, then executes $P$. The term $M$ must be a
function application $f(M_1, \ldots, M_n)$. The function $f$ need not
be declared before. When the process $P$ is absent, nothing is
executed after the event. Events are not really
part of the cryptographic protocol, but are used for authenticity
specifications~\cite{Blanchet2002} and other properties of protocols. 
Events can be used to keep track of which steps of the protocol are executed.

\item The phase separation command $\texttt{phase }n\texttt{;}P$
indicates the beginning of phase $n$. Intuitively, we consider
protocols split in several phases, and the instructions under
$\texttt{phase }n$ are active only during the $n$-th phase of the
protocol.
%
So the process first executes phase 0, that is, it executes all
instructions not under $\texttt{phase }i$ for $i \geq 1$.
%
Then, when changing from phase 0 to phase 1, it discards all processes
that have not reached a $\texttt{phase }i$ instruction for $i \geq 1$
and executes the instructions under $\texttt{phase }1$ but not under
$\texttt{phase }i$ for $i \geq 2$.
%
More generally, when changing from phase $n$ to phase $n+1$, all
processes that have not reached a $\texttt{phase }i$ instruction for
$i \geq n+1$ are discarded and the instructions under $\texttt{phase
}n+1$ but not under $\texttt{phase }i$ for $i \geq n+2$ are executed.
%
The adversary obviously keeps its knowledge when changing phases.

Phases can be used to model scenarios in which temporality is
important, such as when a long-term key is published after some
sessions are executed and we want to determine whether the adversary
can then have the session secrets. (The long-term keys are then
published in phase 1, while the rest of the protocol is in phase 0.)
Similarly, phases can be used to model protocols that reveal a secret
at the end of the session.

The phase number must be at least 1. Phases cannot be used with
key compromise, \texttt{param keyCompromise = approx.}  or
\texttt{param keyCompromise = strict.}, because key compromise
introduces itself a 2-phase process.

\item The synchronization command $\texttt{sync }n\texttt{;}P$
introduces a global synchronization, which has some similarity
with phases. The global synchronizations must be executed in
increasing order. The process waits until all $\texttt{sync }n$
commands are reached before executing the synchronization $n$.
More precisely, assuming $n$ is the smallest synchronization
number that occurs in the initial process and has not been executed yet,
if the initial process contains $k$ commands
$\texttt{sync }n$, then the process waits until it reaches
exactly $k$ commands $\texttt{sync }n$, then it executes
the synchronization $n$ and continues after the $\texttt{sync }n$ 
commands.
So, in contrast to phases, processes are never discarded
by synchronization, but the process may block in case
some synchronizations cannot be reached or are discarded for
instance by a test that fails above them.

The synchronization number must be at least 1. Synchronizations
$\texttt{sync }n$ cannot occur under replications.  Synchronizations
cannot be used with phases or with key compromise. Synchronizations
can help proving equivalences with \texttt{choice}, because they allow
swapping data between processes at the synchronization points.

\end{itemize}

Two cases have to be distinguished:
\begin{itemize}

\item The terms in the process never contain \texttt{choice}. The
process defines one pi calculus process, and we can ask the various
queries (\texttt{query}, \texttt{noninterf}, \texttt{weaksecret}).

\item The terms in the process contain \texttt{choice}. 
The process in fact defines two pi calculus processes:
one process in which the first argument of \texttt{choice} is used,
and one in which the second one is used. The verifier then tries
to show the observational equivalence of these two processes~\cite{Blanchet05}.
The queries \texttt{query}, \texttt{noninterf}, \texttt{weaksecret},
and key compromise cannot be used.
(The keyword \texttt{diff} is also accepted as a synonym for \texttt{choice}.)

\end{itemize}

\section{Output of the system}

The system gives an output of the following form:
\begin{verbatim}
Starting rules:
Rule 10: attacker:c[]
Rule 9: attacker:k -> attacker:host(k)
...
Completing...
Completed rules:
attacker:encrypt(secretB[],k[Kas[],Kbs[],Na[],Nb[host(Kas[]),Na[]]])
attacker:v147 & attacker:v148 -> attacker:encrypt(v148,k[Kas[],v147,Na[],v148])
...
ok, secrecy assumption verified: fact unreachable attacker:Kbs[]
...
goal unreachable: attacker:secretB[]
...
\end{verbatim}
First, it displays the Horn clauses representing the protocol. If you
use the Horn clauses input, these are the clauses you entered. If you
use the pi calculus input, these clauses are the result of a
translation of the process, described in~\cite{Abadi04c}. This
translation uses mainly two predicates $\texttt{attacker:}M$ meaning that the
adversary may have $M$, and $\texttt{mess:}C\texttt{,}M$ meaning that the
message $M$ may be sent on channel $C$. It uses moreover a predicate
$\texttt{end}$ and a blocking predicate $\texttt{begin}$ for
proofs involving events. Some other predicates are used for modeling the
compromise of session keys. The clauses are numbered. These numbers are
used in the following of the output to reference the clauses.

Second, the system completes the clauses, using a resolution-based
algorithm. Depending on the \texttt{verbose} parameter, it prints all
clauses it creates, or only numbers of clauses every 200th clause created.

Third, it outputs the list of clauses obtained after completion (after
the words \texttt{Completed rules}).

Fourth, if you have given secrecy assumptions, using the declaration
\texttt{not}, the system checks them. In case the secrecy assumption
is not satisfied, it stops immediately.

Fifth, the system checks each goal you have given using
\texttt{query}. If the fact mentioned on the left of \texttt{==>}
in the query is derivable, for each found clause that derives it, 
the system outputs \texttt{goal
reachable}, and a derivation of the clause from the initial clauses.  
The derivation is
built according to the following format:
\begin{itemize}
\item When a part of the derivation is done using one of the rules:
\begin{align*}
&\texttt{rule }n\ \nonterm{fact proved by rule $n$}\\
&\quad\nonterm{derivation of first hypothesis of rule $n$}\\
&\quad\ldots\\
&\quad\nonterm{derivation of last hypothesis of rule $n$}
\end{align*}

\item When reusing an already proved fact:
\[\texttt{duplicate }\nonterm{fact}\]
You should look for the derivation of $\nonterm{fact}$ somewhere under the
\texttt{duplicate} line.

\item When taking the $n$-th element of a tuple (In the Horn clauses
input system, this happens when you have declared the predicate $p$ 
\texttt{decompData}. In the pi calculus input system, this happens with $p =
\texttt{attacker}$.):
\begin{align*}
&n\texttt{-th }p:M_n\\
&\quad\nonterm{derivation of $p:(M_1, \ldots, M_k)$}
\end{align*}
A similar situation happens when you have declared $\texttt{data\ }f$:
\begin{align*}
&n\texttt{-th }p:M_n\\
&\quad\nonterm{derivation of $p:f(M_1, \ldots, M_k)$}
\end{align*}

\item When building a tuple (In the Horn clauses input system, this
happens when you have declared the predicate $p$ \texttt{decompData}. In the pi
calculus input system, this happens with $p = \texttt{attacker}$.):
\begin{align*}
&k\texttt{-tuple }p:(M_1, \ldots, M_k)\\
&\quad\nonterm{derivation of $p:M_1$}\\
&\quad\ldots\\
&\quad\nonterm{derivation of $p:M_k$}
\end{align*}
A similar situation happens when you have declared $\texttt{data\ }f$:
\begin{align*}
&f\texttt{-tuple }p:f(M_1, \ldots, M_k)\\
&\quad\nonterm{derivation of $p:M_1$}\\
&\quad\ldots\\
&\quad\nonterm{derivation of $p:M_k$}
\end{align*}

\item When proving $p:x$ (in the Horn clauses input system, when you
have declared the predicate $p$ \texttt{elimVar}; in the pi calculus input system,
when $p = \texttt{attacker}$.)
\[\texttt{any }p:x\]

\end{itemize}
Note that the derivations of inequalities of terms
are omitted.

At the end, it concludes with
\texttt{RESULT Query \ldots is true.} or 
\texttt{RESULT Query \ldots cannot be proved.}.


\bibliography{biblio}
\bibliographystyle{abbrv} %alpha

\end{document}
